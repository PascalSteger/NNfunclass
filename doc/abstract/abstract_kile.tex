\documentclass[a4paper,10pt]{paper}

% Title Page
\title{2D Function Fitting\\with Artificial Intelligence}
\author{Pascal S. P. Steger}

\begin{document}
\maketitle

\begin{abstract}
I present an application of artificial intelligence to function
classification and regression starting from a two-dimensional dataset
of measurements with errors. Three sets of methods are considered:
Neural networks, evolutionary optimization and clustering in Fourier space.

The overall procedure will be shown; starting with the preparation of
the data, the modular parts of the program are explained, sample runs and
runtime characteristics are sketched, limitations and not recognized
functions are shown. A comparison concludes the work.

The focus lies on adaptation of known algorithms to this specific
problem; we restrict ourselves to smooth functions $y=f(x)$.
\end{abstract}

One of the observable properties of galaxies that first spring into mind is
its shape. In order to connect the measured shape of the baryons with the
shape of the more influential dark matter part, we investigate the alignment
of halo shape parameters with each other.

The temporal evolution of the shape is given, as pointed out by
\cite{Valluri2010}, by adaptions of individual particle orbits to the changing
potential. The potential is changed by encounters with neighboring halos or by
interaction with the large scale structure. All changes in particle orbits
affect the angular momentum, which in turn can be observed from the rotation
of the baryonic matter in the center of the halo. In this article we
investigate alignments of shape parameters of components of the halo with each
other as well as with angular momentum and the large scale structure.

It has been established by observations and reproduced in simulations that
there are a number of significant alignments, indeed: In gravitational
potentials of clusters, gas traces the shape of the dark matter halo pretty
well outside the core (\cite{Lau2010}). Further away, it is found that the
major axes of subhalos align with the direction to the center of mass of the
respective host halo (\cite{Yang2006}, \cite{Faltenbacher2007},
\cite{Faltenbacher2008}), indicating a contribution from small scale
gravitational influence. On bigger scales, \cite{Wang2008} found alignments
between the major axes of host galaxies in neighboring groups as well as
impact on the alignments of satellites. \cite{Aragon-Calvo2007} indicate that
the minor axes tend to lie perpendicular to the host structure. On the very
biggest scales, \cite{Basilakos2006} find an alignment between cluster and
supercluster major axes, while \cite{Cuesta2008} report that halos on the
shells of voids preferentially align their angular momentum and shape
perpendicular to the direction joining halo and void center.

%
%% ANGULAR MOMENTUM
% searching connection with dynamical property
% quite basic stuff, check/delete !
In order to understand the evolution of galaxies one has to account for the
buildup of dynamical properties as e.g. angular momentum. \cite{Peebles1969}
showed that the angular momentum observed nowadays agrees with galaxy
formation through gravitational instability. Starting from Gaussian initial
conditions at early time, tidal torques of large scale structures induce
torquing on smaller structures and seed the first buildup of angular momentum,
see e.g. \cite{Barnes1987} for simulations with different initial conditions.

The contributions to angular momentum by dark matter, gas and stars differ
considerably:

Angular momentum of dark matter is mainly conserved; simulations performed by
e.g. \cite{Sharma2005} indicate that roughly half of the dark matter is
rotating in the same sense as baryonic matter, while the other half is
counterrotating.

The fact that a net angular momentum is observed in galaxies suggests that
there exists a process to enhance angular momentum for stars and gas. The
configuration with smallest energy for a selfgravitating system with given
angular momentum is a disk, which is observed in late type galaxies, after a
big amount of internal energy of the gas is radiated away. Stars in elliptical
galaxies have left the disks due to orbit instabilities encountering external
gravitational influences. Violent relaxation then leads to the observed number
density distribution; the angular momentum in the central parts of halos is
increased by transferring some small fraction of the halo mass outwards.

With numeric simulations, one tries to quantize the history of angular
momentum buildup and transformation between the different components.
%
\cite{Sharma2005} investigate the angular momentum disalignment of the gas and
dark matter components of halos in a nonradiative $N$-body/SPH simulation and
find a misalignment

\end{document}
